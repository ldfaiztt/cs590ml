\documentclass[12pt,letterpaper]{article}
\usepackage{amsmath,amsthm,amsfonts,amssymb,amscd}
\usepackage{fullpage}
\usepackage{graphicx}
\usepackage{lastpage}
\usepackage{enumerate}
\usepackage{fancyhdr}
\usepackage{hyperref}
\usepackage{mathrsfs}
\usepackage{xcolor}
\usepackage[margin=3cm]{geometry}
\setlength{\parindent}{0.0in}
\setlength{\parskip}{0.05in}

% Edit these as appropriate
\newcommand\course{STA571/CS590.01}
\newcommand\semester{Spring 2014}                   % <-- current semester
\newcommand\papertitle{Generalized Low Rank Approximations of Matrices}                         % <-- paper title
\newcommand\authoryear{Jieping Ye}
\newcommand\yourname{Matt Dickenson}                % <-- your name
\newcommand\login{mcd31}                            % <-- your NetID
\newcommand\hwdate{Due: 7 March, 2014}           % <-- HW due date


\pagestyle{fancyplain}
\headheight 60pt
\chead{Summary of ``\papertitle''\\ ~\\}
\lhead{\small \yourname\ \texttt{\login}\\\course}
\rhead{\small \hwdate}
\headsep 10pt

\begin{document}

% Read and summarize "Generalized Low Rank Approximations of Matrices" by Jieping Ye.
Ye (2004) introduces a novel approach for computing generalized low rank approximations of matrices (GLRAM). This approach is applicable when the data of interest is a sequence of matrices $A$, which each matrix $A_i \in \mathbb{R}^{r \times c}$ representing one datum (e.g. an image). The goal of the method is to compute a reduced representation of $A$ using two matrices with orthonormal columns, $L \in \mathbb{R}^{r \times \ell_1}$ (the left-side transformation) and $R \in \mathbb{R}^{c \times \ell_2}$ (the right-side transformation), and $n$ matrices $D_i \forall A_i \in A$. The researcher sets $\ell_1$ and $\ell_2$ to control the quality of the approximation and the data compression ratio $nrc \over r \ell_1 + c \ell_2 + n \ell_1 \ell_2$. The desired $L, R$, and $D$ are solutions to the optimization problem:
\begin{eqnarray*}
\text{min} \sum_{i=1}^n || A_i - LD_iR^\intercal ||^2_F
\end{eqnarray*}
that is, to minimize the total squared reconstruction error over the sequence of matrices. 

The approximations of $L$ and $U$ can be computed iteratively by an algorithm that is monotonically decreasing in the root mean squared reconstruction error (RMSRE), where 
\begin{eqnarray*}
\text{RMSRE} &=& \sqrt{\frac{1}{n} || A_i - LD_iR^\intercal ||^2_F}.
\end{eqnarray*}
The total time complexity for computing this approximation is $O(I(r+c)^2 \text{max}(\ell_1, \ell_2)n)$. As long as the entries $A_i$ are ``nearly square'' so that $r \approx c$ and $n<N$, GLRAM will be faster than SVD with its complexity of $O(n^2 N)$. 

% In what circumstances and why is dimensionality reduction useful?
When and why might we employ dimensionality reduction methods such as GLRAM? Typically this becomes a concern when analyzing high-dimensional data (e.g. images, videos) pushes us up against the limits of extant computing power. When there is too much data to hold in memory for analysis, image reduction can compress the size while maintaining close approximations of the original data. More generally, data compression is of interest whenever there may be redundancy in our data (e.g. removing stop words and compressing large text files made up of a known vocabulary). These methods are not without their costs, however. For example, Ye does not theoretically motivate the selection of two key parameters, $\ell_1$ and $\ell_2$, despite the influence they have on the resulting approximation and how this choice may interact with the methods that are subsequently applied to the approximated data. 


% curse of dimensionality
% high-dimnesional data (images, videos) pushing us up against computational limits
% image compression and retrieval
% lots of data -- to much to analyze/hold in memory
% redundancy in data (for GLRAM, redundancy across matrices)

% how does it compare to svd and pca?

% downsides:
% doesn't discuss how to choose l_1=l_2=d
% ''the effect of the value of d''

\end{document}
