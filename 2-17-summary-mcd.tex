\documentclass[12pt,letterpaper]{article}
\usepackage{amsmath,amsthm,amsfonts,amssymb,amscd}
\usepackage{fullpage}
\usepackage{graphicx}
\usepackage{lastpage}
\usepackage{enumerate}
\usepackage{fancyhdr}
\usepackage{hyperref}
\usepackage{mathrsfs}
\usepackage{xcolor}
\usepackage[margin=3cm]{geometry}
\setlength{\parindent}{0.0in}
\setlength{\parskip}{0.05in}

% Edit these as appropriate
\newcommand\course{STA571/CS590.01}
\newcommand\semester{Spring 2014}                   % <-- current semester
\newcommand\papertitle{Bayesian Agglomerative Clustering with Coalescents}                         % <-- paper title
\newcommand\authoryear{Teh, Daume and Roy, NIPS, 2008}
\newcommand\yourname{Matt Dickenson}                % <-- your name
\newcommand\login{mcd31}                            % <-- your NetID
\newcommand\hwdate{Due: 17 February, 2014}           % <-- HW due date


\pagestyle{fancyplain}
\headheight 60pt
\chead{Summary of ``\papertitle''\\ ~\\}
\lhead{\small \yourname\ \texttt{\login}\\\course}
\rhead{\small \hwdate}
\headsep 10pt

\begin{document}

% \noindent \emph{Homework Notes:} 

% Read and Summarize Bayesian Agglomerative Clustering with Coalescents - Teh, Daume and Roy, NIPS, 2008.

Teh, Daume and Roy (2008) introduce a new model for Bayesian agglomerative clustering based on a prior over trees derived from population genetics. The prior, known as Kingman's coalescent, describes genealogies in an evolutionary process. $n$ haploid (single-parent) individuals observed at time $t=0$ are assumed to have a common ancestor at some point {before -infty}. 

% How does this hierarchical clustering model compare to Bayesian Hierarchical Clustering? How does the hierarchical component of this model differ from the Hierarchical Dirichlet Process?






\end{document}
